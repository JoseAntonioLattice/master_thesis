In this thesis we have discussed an extension to the Standard Model by promoting the global U$(1)_{B-L}$ invariance to a local symmetry. We form a new hyper\-charge $Y'$, by taking a linear combination of the weak hypercharge $Y$, from the U$(1)_Y$ gauge symmetry, and the  difference between the baryon and lepton numbers $B-L$, from U$(1)_{B-L}$. Therefore, we introduced a new gauge coupling $h'$ that couples the Abelian gauge field to $B-L$.
The new charge $Y'$  is defined as
\begin{equation}
	Y' \equiv 2hY+\frac{h'}{2}(B-L),
\end{equation}
 where $h$ is the coupling to the weak hypercharge. Therefore, the new gauge group is U$(1)_{Y'}$.
 
This way, we converted the Standard Model symmetry group to
 \begin{equation}
	\text{SU}(3)\times \text{SU}(2)_L\times \text{U}(1)_{Y'}.
\end{equation}
 
The introduction of the new gauge coupling generates gauge anomalies, as seen in the triangular diagram in Figure \ref{fig:anomaly}. In order to remove the anomalies, we introduced a right-handed neutrino $\nu_R$ in each fermion genera\-tion. 
 %Therefore, we need the addition of three new fields, a gauge field $\mathcal{A}_{\mu}$ responsible for the local U$(1)_{Y'}$ symmetry, however, gauge anomalies arise, so we also added a right-handed neutrino $\nu_{R}$, and finally we added a new Higgs field $\chi$ in order to give mass to $\nu_{R}$.
We give mass to the neutrinos via the Higgs mechanism. However, if we want to give another mass to the right-handed neutrino, independently from the left-handed neutrino, the standard Majorana term is not applicable, so we introduce a new Higgs field $\chi\in \mathbb{C}$ and its vacuum expectation value $v'$. This new scalar field has a charge $B-L=-2$ in order to preserve gauge invariance. We also introduced a coupling term between the two Higgs fields, $\kappa \Phi^{\dagger}\Phi \chi^{*}\chi$, in the Lagrangian.

As a simplified model, we only considered the U$(1)_{Y'}$ as the gauge group. This model allows for vortex-line solutions or cosmic strings.
 
 We then studied the system of field equations for $\Phi$, $\chi$ and $\mathcal{A}_{\mu}$. We made cylindrical ansätze for these fields in order to study cosmic string solutions and solved the system of equations with appropriate boundary conditions at $r=0$ and $r\to\infty$.

We used $v=246 \ \text{GeV}$ to convert all quantities into physical units. For instance, in some plots of Chapter \ref{chap:results} we used $v=0.5$. This way, the typical length $r = 1$ is equal in physical units to $r = \frac{0.5}{246\ \text{GeV}}0.197\ \text{GeV}\ \text{fm}\approx 0.0004\ \text{fm}$.

We observed that the $\kappa$ value has a modest effect on the solutions for $a$ and $\xi$, but a significant effect on $\phi$, specially in the low $r$ regime.

Our contribution to the literature is the finding of co-axial and overshoot solutions. We found co-axial solutions for the field $\phi$ that are negative at low $r$, pass the $r$ axis, and then approach their positive vacuum expectation value. Surprisingly, we observed also the opposite effect: inside the string, depending on the values of the winding numbers, the fields $\Phi$ or $\chi$ can overshoot its vacuum expectation value. That is, the field can take a higher value than its VEV. The overshoot is more visible when considering the constraints form the SO(10) model, a Grand Unified Theory. Physically, we can interpret the overshoot as a temporarily increase in the mass of a near passing particle.

This modest but fully consistent extension of the Standard Model allows  for a non-standard type of cosmic strings. The tension of these kinds of strings is of the order of $10^{10}\ \text{MeV}^2$, and have a gravitational coupling near to $10^{-30}$.

The scenario of cosmic strings that we have studied seems realistic. The proposals for an observational test are slim, unfortunately. However, this also means that such cosmic strings are most likely not ruled out based on existing data. A valid argument in its favor is the original motivation of explaining why the $B-L$ invariance is an exact symmetry.


