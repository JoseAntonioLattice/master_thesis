\documentclass[12pt,a4paper,twoside]{book}
\usepackage[utf8]{inputenc}
\usepackage[english]{babel}
\usepackage{amsmath}
\usepackage{amsfonts}
\usepackage{amssymb}
\usepackage{graphicx}
\let\tmp\oddsidemargin
\let\oddsidemargin\evensidemargin
\let\evensidemargin\tmp
\reversemarginpar

\author{José Antonio García-Hernández}
\linespread{1.5}
\begin{document}

\begin{titlepage}
\begin{center}

		\includegraphics[width=4.4cm]{Escudo-UNAM.pdf}	\\
		\vspace{0.3cm}	
		\sc			
		\textbf{ UNIVERSIDAD NACIONAL AUT\'ONOMA DE M\'EXICO} \\
		\small Posgrado en Ciencias F\'isicas \\
		\vspace*{0.3cm}
		\begin{tabular}{c}
		\hline\hline
\vspace{-5mm}\\
		\Huge{\sc{The profile of non-standard}}\\
		\Huge{\sc{cosmic strings}}\\\hline\hline
		\end{tabular}\\
		\vspace*{0.3cm}
		\normalsize
		\textbf{TESIS}\\
		Que para optar por el grado de:\\
Maestro en Ciencias (F\'isica)\\
\vspace*{0.3cm}
\textbf{PRESENTA}\\
{\large Jos\'e Antonio Garc\'ia Hern\'andez}\\
\vspace*{0.3cm}
\textbf{TUTOR PRINCIPAL}\\
Dr.\ Wolfgang Peter Bietenholz\\
Instituto de Ciencias Nucleares\\
\vspace*{0.3cm}
\textbf{COMIT\'E TUTOR}\\
Dr.\ Eduardo Peinado Rodríguez\\
Instituto de F\'isica\\
Dr.\ Jos\'e Alejandro Ayala Mercado \\
Instituto de Ciencias Nucleares\\

	\vspace*{0.3cm}	
		{Ciudad Universitaria, Cd. de M\'exico, Abril 2023}
		
\end{center}

\end{titlepage}

\end{document}