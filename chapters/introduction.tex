In recent years, the study of topological defects is becoming of great interest in modern physics. In field theories, topological defects could be found as solitonic stable solutions of the classical field equations.

In general, a topological defect can be defined as a discontinuity in the order parameter space of the system. In condensed matter physics, there are several examples of topological defects, but in the physics of the early universe, they are still hypothetical. For instance, topological defects are a generic prediction in Grand Unified Theories.

It is generally assumed that in the early universe several phase transitions occurred, giving rise to topological defects by means of the Kibble mechanism \cite{Kibble1976,Kibble1980}. If they exist, we expect on average at least one topological defect per horizon volume.

An example of a topological defect forming in the early universe is the vortex line or \emph{cosmic string}. Cosmic strings are elongated concentrations of energy that are very thin, and can be considered effectively as 1-dimensional. They can be closed or open and very large, of the order of a cosmic horizon.

If cosmic strings exist, they can be of various types. For instance, they can be global strings, which emerge from a global symmetry, or local strings, which originate from a local symmetry. Global strings have the property that their tension is infinite. On the other hand, local strings have a finite tension. U(1) local strings are also known as Nielsen-Olesen strings, and have the property that their magnetic flux is quantized. 

In addition, there is another type of string called the superconducting string \cite{WITTEN1985557}. This type of string behaves like a superconducting wire, in which current can be carried by bosons or fermions.


Intense research of cosmic strings is being performed recently, in particular, some research like in Ref.\ \cite{PhysRevD.97.102002} aims to detect their gravitational waves signals, which set constraints to the tension of the string. Moreover, cosmic strings have a characteristic discontinuity effect in the CMB (Cosmic Microwave Background) temperature which could possibly be measured. Cosmic super\-strings, on the other hand can produce cosmic rays such as $\gamma$-rays.

Since they are very thin, of the order of 1 fm or even less, their dynamics  can be studied in the zero width limit as Nambu-Goto strings. 

They were once believed to be the seed for large structures such as galaxies, as reviewed in Ref.\ \cite{kibble1986}. However, measurements of the CMB power spectrum by COBE  (COsmic Background Explorer) and WMAP (the Wil\-kin\-son Microwave Anisotropy Probe) discarded the possibility of cosmic strings having an effect on the formation of large structures. These meas\-ure\-ments showed that the angular power spectrum has acoustic peaks that are not explained by cosmic strings, see Ref.\ \cite{pogo2006}. 

The importance of topological defects in the universe is that they would be directly observables as relics of the primordial fields in the early universe.

The objective of this thesis is to study numerically the profile of cosmic strings related to the local invariance of the symmetry $\text{U}(1)_{B-L}$. We are particularly interested in co-axial string solutions. Co-axial solutions are negative at low $r$, pass the $r$ axis, and then approach their positive boundary value at $r\to\infty$.

Throughout this thesis, we will work with natural units where $c=\hbar=k_B = 1$.

In Chapter 2, we present the mathematical background required, such as concepts in topology and group theory, and also give examples of topological defects in other branches of physics. In Chapter 3, we focus on cosmic strings, we study the solutions to global and local U(1) strings, and give a review of the research done to detect them. Chapter 4 is completely original, we describe cosmic strings solutions enabled by a scenario ``Beyond the Standard Model" (BSM) where we promote the global U$(1)_{B-L}$ symmetry to a local symmetry and we combine it with U$(1)_Y$ which leads to the group we call U$(1)_{Y'}$. In Chapter 5, we present solutions to the equations of motion for the fields and discuss the profile of local U$(1)_{Y'}$ cosmic strings within this BSM model. 


