\ESP{Se asume que a través del llamado \textit{mecanismo de Kibble} en el universo temprano pudieron haberse formado defectos topológicos. En esta tesis discutimos un modelo más allá del Modelo Estándar que permite un tipo de defectos topológicos llamado \textit{cuerda cósmica}. Para estudiar soluciones de cuerdas cósmicas, primero promovemos la simetría global U$(1)_{B-L}$ a una simetría local y agregamos un nuevo acoplamiento de norma. La cancelación de las anomalías de norma se logra agregando un neutrino derecho a cada generaci\'on de leptones. Además, se agrega un nuevo campo de Higgs para dar masa al neutrino derecho. Finalmente, estudiamos las ecuaciones de movimiento de los dos campos de Higgs y el campo de norma, con el fin de obtener los perfiles de las cuerdas cósmicas. Particularmente, descubrimos un tipo de soluciones a las que llamamos  de cuerdas cósmicas coaxiales. Adicionalmente, obtuvimos que la tensión de las cuerdas es del orden de $10^{19}\ \text{GeV}^{2}$ y su acoplamiento gravitacional $\sim 10^{-30}$ el cual está por debajo de las constricciones obtenidas por la colaboración LIGO.}